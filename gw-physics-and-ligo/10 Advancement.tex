\subsection{Advancements in LIGO}
With the aim to detect the smallest of gravitational waves from the farthest of places in the universe, the LIGO team strived harder to perfect the LIGO setup. They made a myriad of changes to LIGO, and a few are discussed in this section.

\subsubsection{Enhanced LIGO}
After the end of the Science Run 5 in 2007, a few changes and upgrades were made to the initial LIGO, which enhanced its performance dramatically and was, hence, dubbed Enhanced LIGO. Some of the upgrades include:
\begin{itemize}
\item Increased laser power

\item Homodyne detection\footnote{Homodyne detection is a method of extracting information encoded as modulation of the phase of an oscillating signal, by comparing with a signal of standard oscillation that would be the same as the signal, if it carried null information. \cite{Homodyne}}

\item Output-mode cleaner

\item In-vacuum readout hardware
\end{itemize}
At the most sensitive frequencies, the strain sensitivity was improved by approximately 30\%, which helped the instrument's root-mean-square strain noise reach an unprecedented level of $2\times 10^{-22}$ in a 100 Hz band.\cite{Adv_LIGO_Paper}\\
After the changes, Science Run 6 began in 2009 and concluded in 2010. Even though no GWs were detected during this run, there were some interesting astrophysical results obtained, including gaining some insight on the upper limit on stochastic GW background of cosmological origin. \cite{Adv_LIGO}

\subsubsection{Advanced LIGO}
After Science Run 6 ended in 2010, LIGO went offline for quite a few years for a major upgrade, installing new second-generation detectors in both the infrastructures. To increase the interaction time with a gravitational wave, just like in Initial LIGO, Fabry-P\'erot cavities were used in the arms, and to increase the effective laser power, power recycling was used. With the help of signal recycling, the frequency response of Advanced LIGO was improved, increasing the design strain sensitivity in the most sensitive frequency region (around 100 Hz) to a factor of 10 in comparison to Initial LIGO. Since the gravitationally probed volume of the universe is directly proportional to the cube of the strain sensitivity, an increase by a factor of 10 means that there is an enormous increase in the number of potential astrophysical sources waiting to be detected by these instruments.\\

By replacing every interferometer components with improved technologies, and by employing much better seismic isolation and test mass suspension, the low frequency end of the sensitivity band was moved from 40 Hz to 10 Hz. At mid and high frequencies, higher laser power, larger test masses and improved mirror coatings helped improve the sensitivity. \cite{Adv_LIGO_Paper}\\ 

These and other changes helped Advanced LIGO operate at a sensitivity roughly 3 times more than initial LIGO. This project attracted a lot of participants, including the Australian National University and University of Adelaide who made significant contribution to the project. By the time it was ready to start its first operation in September 2015, the LIGO team was composed of more than 900 scientists worldwide. \cite{Adv_LIGO}\\

Advanced LIGO has the exciting potential to detect hyperbolic encounters of primordial black holes (PBH). Such encounters will have shapes similar to "tear drop glitch" shapes, and will look like bursts with characteristic frequency at peak strain amplitude. Advanced LIGO has reported such events, and even though accidental noise in the detectors were given credit for that, those events being PBH hyperbolic encounters is a possibility worth exploring. And if these events are PBH encounters,then they can be used to obtain valuable information about PBH velocity,mass and their spatial distribution. \cite{Garc_a_Bellido_2017}

\subsection{Gravitational Wave Interferometers around the World}
Other than LIGO and Advanced LIGO, there are and were other GW Interferometers around the world. These interferometers are grouped into generations based on the technology they used. The first generation interferometers were deployed in the 1990s and the 2000s and were the ones to lay down the foundations of the important technologies. The second generation of detectors improved upon the previous ones and ran in the 2010s. They introduced new technologies cryogenic mirrors and injection of squeezed vacuum, with the help of which the GW event GW150914 was detected. The third generation of detectors are still in the planning phase, and aim to improve the technologies of the previous generation detectors. \cite{GWO}

\subsubsection{First Generation}
\begin{itemize}
\item \textbf{TAMA 300} - This GW detector is located at the Mitaka campus of the National Astronomical Observatory of Japan. It is a project of the GW studies group at the Institute of Cosmic Ray Research (ICRR) of the University of Tokyo. The project's construction start in 1995, and data was collected from 1999 to 2004. \cite{TAMA_300}

\item \textbf{Virgo} - A scientific collaboration of six different countries, Virgo Interferometer is located in Santo Stefano a Macerata, near the city of Pisa, Italy. Its arms are a kilometer shorter than LIGO's arms. The construction was completed in 2003, and initial data was taken from 2007 to 2011 during four science runs. \cite{Virgo}
\end{itemize}

\subsubsection{Second Generation}
\begin{itemize}
\item \textbf{Advanced Virgo} - Not being able to detect any GWs, the initial Virgo was decommissioned in 2011 and replaced by Advanced Virgo, which is 10 times more sensitive than initial LIGO, had larger mirrors, and improved optical performances. It started commissioning in 2016, and in August 2017, along with Advanced LIGO, detected a GW signal GW170814. \cite{Virgo}

\item \textbf{LIGO-India} -  This is a planned advanced GW interferometer, which will be identical to the advanced LIGO interferometers in USA. Spearheaded by a consortium of Indian gravitational-wave physicists (IndiGO), the inception of the planning phase of the project occurred way back in 2009. A site near Aundha Nagnath in the Hingoli District, Maharashtra has been selected. The aim of this project is to join a network of detectors throughout the world, which will help in increasing the number of sources of GWs detected. \cite{LIGO_India}
\end{itemize}

\subsubsection{Third Generation}
\textbf{Einstein Telescope (ET)} - Currently under study by a few institutions in the European Union, the ET aims to test Einstein's General Theory of Relativity in strong field conditions and increase the precision of gravitational wave astronomy. The current second generation detectors lack the technologies to realize precision gravitational wave astronomy, especially with the likes of massive stellar bodies and highly asymmetric (in mass) binary systems. To increase precision and circumvent any technological problem posed by the second generation detectors, new infrastructure will have to be implemented: ET will be an underground detector to reduce seismic noise and cryogenic facilities will be used to cool down the mirrors. \cite{ET}

\subsubsection{Space based Observatories}
\textbf{Laser Interferometer Space Antenna (LISA)} - A joint effort between NASA and the ESA, the LISA will constitute a constellation of three spacecrafts arranged in an equilateral triangle with sides of about 2.5 million kms long, flying along a heliocentric orbit. The entire arrangement will be ten times larger than the orbit of the moon and will be placed at the same distance from the Sun as the Earth, but 20 degrees behind Earth. \cite{LISA}
\pagebreak
